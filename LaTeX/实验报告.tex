\documentclass[a4paper,12pt]{article}
\usepackage{ctex}
\title{采购试验一}
\author{刘煜}

\begin{document}
        
    \subsection{案件综述}
    

    被告人李某为牟取非法利益,预谋以修改大型互联网网站域名解析指向的方法,劫持互联网流量访问相关赌博网站,获取境外赌博网站广告推广流量提成。

    2014年10月20日,李某冒充某知名网站工作人员,采取伪造该网站公司营业执照等方式,骗取该网站注册服务提供商信任,获取网站域名解析服务管理权限。10月21日,李某通过其在域名解析服务网站平台注册的账号,利用该平台相关功能自动生成了该知名网站二级子域名部分DNS(域名系统)解析列表,修改该网站子域名的IP指向,使其连接至自己租用境外虚拟服务器建立的赌博网站广告发布页面。当日19时许,李某对该网站域名解析服务器指向的修改生效,致使该网站不能正常运行。23时许,该知名网站经技术排查恢复了网站正常运行。11月25日,李某被公安机关抓获。至案发时,李某未及获利。

    经司法鉴定,该知名网站共有559万有效用户,其中邮箱系统有36万有效用户。按日均电脑客户端访问量计算,10月7日至10月20日邮箱系统日均访问量达12.3万。李某的行为造成该知名网站10月21日19时至23时长达四小时左右无法正常发挥其服务功能,案发当日仅邮件系统电脑客户端访问量就从12.3万减少至4.43万。
    
    \subsection{诉讼过程和结果}
    被告人李某违反国家规定,对计算机信息系统功能进行修改,造成计算机信息系统不能正常运行,后果特别严重,其行为已构成破坏计算机信息系统罪,应予处罚。公诉机关的指控成立。

    被告人李某到案后如实供述自己的罪行,依法予以从轻处罚。根据被告人犯罪的事实、性质、情节和对于社会的危害程度,依照《中华人民共和国刑法》第二百八十六条第一款、第六十七条第三款、《最高人民法院、最高人民检察院关于办理危害计算机信息系统安全刑事案件应用法律若干问题的解释》第四条第二款第二项之规定,判决如下:


    被告人李某犯破坏计算机信息系统罪,判处有期徒刑五年。

    \subsection{案件分析}
    很常见的黑产手法,被告人李某为牟取非法利益,以修改大型互联网网站域名解析指向的方法,劫持互联网流量访问相关赌博网站,获取境外赌博网站广告推广流量提成,犯破坏计算机信息系统罪,判处有期徒刑五年。

    修改目标网站域名解析服务器,目标网站域名被恶意解析到其他IP地址,无法正常发挥网站服务功能,这种行为实质是对计算机信息系统功能的修改、干扰,符合刑法第二百八十六条第一款“对计算机信息系统功能进行删除、修改、增加、干扰”的规定。



\end{document}